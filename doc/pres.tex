\documentclass[aspectratio=43]{beamer}
\usetheme{Berlin}

\usepackage[czech]{babel}
\usecolortheme{dolphin}
\definecolor{accent}{HTML}{559915}
\setbeamercolor{structure}{fg=accent}
\usepackage{graphicx}
\usepackage{dirtree}
\usepackage{listings}
\usepackage[T1]{fontenc}
\usepackage{lmodern}
\usepackage[utf8]{inputenc}
\usepackage{caption}
\usepackage{bbding}
\usepackage{xurl}
\usepackage{scrextend}
\usepackage{minted}
\usepackage{appendixnumberbeamer}

\captionsetup{labelformat=empty}

\beamertemplatenavigationsymbolsempty
\defbeamertemplate*{title page}{customized}[1][]
{
	\begin{center}
	\usebeamerfont{title}\inserttitle\par
	\usebeamerfont{subtitle}\insertsubtitle\par
	\end{center}
	\vfill
	\usebeamercolor[fg]{subtitle}
	\usebeamerfont{author}\insertauthor\par
	\usebeamerfont{institute}\insertinstitute\par
	\usebeamerfont{date}\insertdate\par
	\usebeamercolor[fg]{titlegraphic}\inserttitlegraphic
}

\hypersetup{unicode}
\hypersetup{breaklinks=true}


\title{Scheduler}
\subtitle{kalendář a budík v jednom}
\author{Josef Litoš}
\date{}
\institute{ČVUT -- FIT}
\setbeamertemplate{sidebar right}{}
\setbeamertemplate{footline}{%
\hfill\textbf{\insertframenumber{} / \inserttotalframenumber} \hspace{0.01cm} \vspace{0.1cm}}
\setbeamerfont{footnote}{size=\tiny}

\begin{document}

\begin{frame}[plain]
	\maketitle
\end{frame}

\clearpage
\setcounter{framenumber}{0}

\frame{
	\frametitle{Vznik}
	\section{Vznik}
	\begin{itemize}
		\item je rozlišení nutné
		\item systematické složení
		\item prozkoumání modernějších přístupů
	\end{itemize}
}

\section{Struktura}
\frame{
	\frametitle{Komunikace s databází}
	\begin{itemize}
		\item pouze primitivní
		\item \textit{room} anotace odvádí práci za nás
		\item strukturu nutné změnit při hlubším rozšíření funkcí
		\item objekty řeší pouze sebe, změny delegují dál
	\end{itemize}
}

\frame{
	\frametitle{Nastavení}
	\begin{itemize}
		\item podpora samotným systémem
		\item existují 3+ api, fragmentace
		\item vyzkoušení podoby v \textit{SimpleTools}
	\end{itemize}
}

\section{Provoz}
\frame{
	\frametitle{Načtení}
	\begin{itemize}
		\item vyhnutí čekání na data
		\item po načtení celé databáze teprve pracujeme s daty
		\item nezávislé na typu spuštění (nastavení/celá aplikace)
	\end{itemize}
}

\frame{
	\frametitle{Běh}
	\begin{itemize}
		\item registrace nejbližšího oznámení
		\item nová při každé změně
		\item předchozí zabíjíme, pouze poslední zůstane
		\item mohlo by být robustnější
	\end{itemize}
}

\section{Tvorba}
\frame{
	\frametitle{Editor}
	\begin{itemize}
		\item všechna práce uživatele
		\item možná příliš pevně spojené
		\item pouze sbírá data, neřeší zpracování
		\item zjištění: systémové dialogy pomalé?
	\end{itemize}
}

\frame{
	\frametitle{Typy oznámení}
	\begin{itemize}
		\item řazení datem
		\item kontrola stavu
		\item zrušení možné obousměrně
		\item automatické skrytí ostatních
	\end{itemize}
}

\frame{
	\frametitle{Závěr}
	\section{Závěr}
	\begin{itemize}
		\item plně funkční
		\item nutno přidat hook pro coldboot
		\item hlubší průzkum oznámení a synchronizace
		\item UX potřebují zkrášlit
		\item vytvořit logo
	\end{itemize}
}
\end{document}

